\documentclass{ctexart}

\title{作业二:用户输入实验报告}


\author{作者姓名:金文宽 \\作者专业和学号:统计学3210105908}

\begin{document}

\maketitle


尝试了用户输入的小实验,并将展示内容进行了一些有趣的修改。
\section{实验内容}
Shell脚本如下:
\begin{verbatim}
#!/bin/bash

echo "Is 91 a prime number?Please answer yes or no."
read answer

case "$answer" in
     yes) echo "Stupid!91=7*13";;
     no) echo "Wow,you are right!91=7*13";;
     y) echo "Stupid!91=7*13";;
     n) echo "Wow,you are right!91=7*13";;
     *) echo "I guess you should retake your primary education!";;
esac

exit 0
\end{verbatim}
输出结果:
\verb|Is 91 a prime number?Please answer yes or no.|

\textbf{yes}

\verb|Stupid!91=7*13|
\section{学习收获}
在上面的用户输入实验中,我学到了\verb|case|结构,当\verb|case|语句被执行时,他就会把变量\verb|answer|的内容与下面的这些字符串进行比较,并且是依次进行的,一旦匹配成功,那么就会立刻执行后面的代码,并且结束比较.同时还学到了一个小技巧,那就是在最后加上一个\verb|*|,表示如果前面的都没有匹配成功,那么就执行\verb|*|后面的代码.这里的原理就是\verb|*|代表anything,任何字符串,并且他的比较是依次做下来的,这也是\verb|*|要放到最后的原因,倘若将\verb|*|放在最前面,那么其他的情况也就不会执行了,只会始终执行\verb|*|后面的代码.

原来代码的问题是询问时间然后输出早上好或者晚上好,我进行了一个有趣的改动,将问题改成了一个常人容易犯错误的小问题:91是否是素数?如果你现在依然认为91是素数的话...

\verb|I guess you should retake your primary education!|

\end{document}
